% Use this template to write your solutions
\documentclass[12pt]{article}
\usepackage{fancyhdr}
\usepackage{amsmath}
\usepackage{amssymb}
\usepackage{relsize}
\usepackage{graphicx}
\graphicspath{ {images/} }

\usepackage{courier}

\usepackage[usenames,dvipsnames]{color} % Required for specifying custom colors and referring to colors by name
\usepackage[pdftex]{hyperref} % For hyperlinks in the PDF
\hypersetup{
  colorlinks=true,
  linkcolor=MyBlue, 
  citecolor=MyRed,
  urlcolor= MyBlue
}

\definecolor{MyRed}{rgb}{0.99, 0.0, 0.0} 
\definecolor{MyGreen}{rgb}{0.0,0.4,0.0} 
\definecolor{MyBlue}{rgb}{0.0, 0.0, 0.6}



% Set the margins
%
\setlength{\textheight}{8.5in}
\setlength{\headheight}{.2in}
\setlength{\headsep}{.25in}
\setlength{\topmargin}{0in}
\setlength{\textwidth}{6.5in}
\setlength{\oddsidemargin}{0in}
\setlength{\evensidemargin}{0in}



% Macros
\newcommand{\myN}{\hbox{N\hspace*{-.9em}I\hspace*{.4em}}}
\newcommand{\myZ}{\hbox{Z}^+}
\newcommand{\myR}{\hbox{R}}
\newcommand{\txt}[1]
{\textnormal{#1}}

\newcommand{\prob}[1]
{\textbf{P}\{{#1}\}}

\newcommand{\E}[1]
{\textbf{E}[{#1}]}

\newcommand{\mi}[1]
{\txt{min}_{#1}}

\newcommand{\ma}[1]
{\txt{max}_{#1}}

\newcommand{\myfunction}[3]
{${#1} : {#2} \rightarrow {#3}$ }

\newcommand{\myzrfunction}[1]
{\myfunction{#1}{{\myZ}}{{\myR}}}

% changing lists to (a), (b), (c) instead of 1. 2. 3. 
\renewcommand{\labelenumi}{(\alph{enumi})}

% Formating Macros

\newcommand{\myheader}[2]
{\vspace*{-0.5in}
\noindent
{#1} 

\noindent
{#2} 
\noindent
}  % end \myheader 

\newcommand{\courseheader}
{
\bf{COS 598C Spring 2015}
}

\newcommand{\authorheader}
{
Kiran Vodrahalli and Yanchen Wang
}

% Running head (goes at top of each page, beginning with page 2.
% Must precede by \pagestyle{myheadings}.
\newcommand{\myrunninghead}[2]
{\markright{{\bf{#1}: \txt{#2}}}}

\newcommand{\kiranhead}
{
 \lhead{\courseheader}
 \rhead{\authorheader}
}


\newcommand{\mytitle}[1]
{\begin{center}
{\large {\bf {#1}}}
\end{center}}

\newcommand{\probnum}[1]
{\large {\bf Problem {#1}}}

\newcommand{\mysection}[1]
{\noindent {\bf {#1}}}

%%%%%% Begin document with header and title %%%%%%%%%%%%%%%%%%%%%%%%%


\begin{document}
\pagestyle{fancy}

\kiranhead

\probnum{1}

The goal in this problem will be to explore methods of combining bounding box proposals to classify objects and form predictions about where they are located in a picture. We will be using the Overfeat paper as well as the SUN09 dataset. A set of images from the SUN09 dataset has been provided in your starter code. 

\begin{enumerate}

\item First, download a set of object proposals from \href{http://people.csail.mit.edu/myungjin/HContext.html}{$\txt{HContext.html}$}: You will need the \href{http://groups.csail.mit.edu/vision/Hcontext/data/detectorOutputsText.tar.gz}{detectorOutputsText.tar.gz} file. First you will need to gzip and untar the folders. These files are organized into two folders, \texttt{train} and \texttt{test}, each with subdirectories each specifying an object. Each of these object directories contains many .txt files named after the image to which the object belongs. Each line in the file contains bounding box locations and scores for that image in the format $[x_1 \txt{ } y_1 \txt{ } x_2 \txt{ } y_2 $\texttt{ score}$]$. 

Your first task is to write a Python script transform this directory into a directory that instead contains a directory for each image in the image set provided, with a set of files for each image containing bounding box information. These files should be labeled $\{$\texttt{object name}$\}$.txt. Note: The \texttt{os} module is necessary for this task. Another useful module may be the \texttt{shutil} module.

[Justification for the problem: Sometimes in research and data analysis, datasets are not provided in a nice format and we have to reformat them. Learning how to use useful tools like Python for this sort of task is a useful skill for any researcher who works with data.]

\item For each provided image, implement parts $(d)-(g)$ of the greedy merge algorithm for the proposed bounding boxes, as described on pages $10-11$ of the \href{http://arxiv.org/pdf/1312.6229v4.pdf}{Overfeat paper}. You will need to use the bounding box data from both the \texttt{train} and \texttt{test} folders. In the paper, they discuss the scale $s$: ignore this aspect of the algorithm, and assume that the bounding boxes of the same class for each image are all the proposals that exist. Note that we also are going to assume at first that there is one instance of an object in the image. Be sure to explain and justify your implementation of \texttt{match$\_$score}, as only the idea of this algorithm is provided in the paper. Also be sure to assign a confidence score to the resulting merged bounding boxes based on the confidences of the individual bound scores (read pg. $11$ closely). 

[Justification for the problem: It is important to be able to read a paper and implement algorithms from a simple, sometimes incomplete description and evaluate it for yourself. This question is a simple exercise in implementing a simple algorithm where some information was left out, requiring the reader of the paper to think about and fill in the gaps. ]

\item 

\item

\item

\end{enumerate}

\probnum{2}

\begin{enumerate}

\item

\item

\item

\end{enumerate}

\probnum{3}

\begin{enumerate}

\item

\item

\item

\end{enumerate}


\end{document}
